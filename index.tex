% Options for packages loaded elsewhere
% Options for packages loaded elsewhere
\PassOptionsToPackage{unicode}{hyperref}
\PassOptionsToPackage{hyphens}{url}
\PassOptionsToPackage{dvipsnames,svgnames,x11names}{xcolor}
%
\documentclass[
  letterpaper,
  sfsidenotes]{tufte-book}
\usepackage{xcolor}
\usepackage{amsmath,amssymb}
\setcounter{secnumdepth}{-\maxdimen} % remove section numbering
\usepackage{iftex}
\ifPDFTeX
  \usepackage[T1]{fontenc}
  \usepackage[utf8]{inputenc}
  \usepackage{textcomp} % provide euro and other symbols
\else % if luatex or xetex
  \usepackage{unicode-math} % this also loads fontspec
  \defaultfontfeatures{Scale=MatchLowercase}
  \defaultfontfeatures[\rmfamily]{Ligatures=TeX,Scale=1}
\fi
\usepackage{lmodern}
\ifPDFTeX\else
  % xetex/luatex font selection
  \setmainfont[]{ETbb}
  \setsansfont[Scale=MatchUppercase]{TeX Gyre Heros}
\fi
% Use upquote if available, for straight quotes in verbatim environments
\IfFileExists{upquote.sty}{\usepackage{upquote}}{}
\IfFileExists{microtype.sty}{% use microtype if available
  \usepackage[]{microtype}
  \UseMicrotypeSet[protrusion]{basicmath} % disable protrusion for tt fonts
}{}
% Make \paragraph and \subparagraph free-standing
\makeatletter
\ifx\paragraph\undefined\else
  \let\oldparagraph\paragraph
  \renewcommand{\paragraph}{
    \@ifstar
      \xxxParagraphStar
      \xxxParagraphNoStar
  }
  \newcommand{\xxxParagraphStar}[1]{\oldparagraph*{#1}\mbox{}}
  \newcommand{\xxxParagraphNoStar}[1]{\oldparagraph{#1}\mbox{}}
\fi
\ifx\subparagraph\undefined\else
  \let\oldsubparagraph\subparagraph
  \renewcommand{\subparagraph}{
    \@ifstar
      \xxxSubParagraphStar
      \xxxSubParagraphNoStar
  }
  \newcommand{\xxxSubParagraphStar}[1]{\oldsubparagraph*{#1}\mbox{}}
  \newcommand{\xxxSubParagraphNoStar}[1]{\oldsubparagraph{#1}\mbox{}}
\fi
\makeatother

\usepackage{color}
\usepackage{fancyvrb}
\newcommand{\VerbBar}{|}
\newcommand{\VERB}{\Verb[commandchars=\\\{\}]}
\DefineVerbatimEnvironment{Highlighting}{Verbatim}{commandchars=\\\{\}}
% Add ',fontsize=\small' for more characters per line
\usepackage{framed}
\definecolor{shadecolor}{RGB}{241,243,245}
\newenvironment{Shaded}{\begin{snugshade}}{\end{snugshade}}
\newcommand{\AlertTok}[1]{\textcolor[rgb]{0.68,0.00,0.00}{#1}}
\newcommand{\AnnotationTok}[1]{\textcolor[rgb]{0.37,0.37,0.37}{#1}}
\newcommand{\AttributeTok}[1]{\textcolor[rgb]{0.40,0.45,0.13}{#1}}
\newcommand{\BaseNTok}[1]{\textcolor[rgb]{0.68,0.00,0.00}{#1}}
\newcommand{\BuiltInTok}[1]{\textcolor[rgb]{0.00,0.23,0.31}{#1}}
\newcommand{\CharTok}[1]{\textcolor[rgb]{0.13,0.47,0.30}{#1}}
\newcommand{\CommentTok}[1]{\textcolor[rgb]{0.37,0.37,0.37}{#1}}
\newcommand{\CommentVarTok}[1]{\textcolor[rgb]{0.37,0.37,0.37}{\textit{#1}}}
\newcommand{\ConstantTok}[1]{\textcolor[rgb]{0.56,0.35,0.01}{#1}}
\newcommand{\ControlFlowTok}[1]{\textcolor[rgb]{0.00,0.23,0.31}{\textbf{#1}}}
\newcommand{\DataTypeTok}[1]{\textcolor[rgb]{0.68,0.00,0.00}{#1}}
\newcommand{\DecValTok}[1]{\textcolor[rgb]{0.68,0.00,0.00}{#1}}
\newcommand{\DocumentationTok}[1]{\textcolor[rgb]{0.37,0.37,0.37}{\textit{#1}}}
\newcommand{\ErrorTok}[1]{\textcolor[rgb]{0.68,0.00,0.00}{#1}}
\newcommand{\ExtensionTok}[1]{\textcolor[rgb]{0.00,0.23,0.31}{#1}}
\newcommand{\FloatTok}[1]{\textcolor[rgb]{0.68,0.00,0.00}{#1}}
\newcommand{\FunctionTok}[1]{\textcolor[rgb]{0.28,0.35,0.67}{#1}}
\newcommand{\ImportTok}[1]{\textcolor[rgb]{0.00,0.46,0.62}{#1}}
\newcommand{\InformationTok}[1]{\textcolor[rgb]{0.37,0.37,0.37}{#1}}
\newcommand{\KeywordTok}[1]{\textcolor[rgb]{0.00,0.23,0.31}{\textbf{#1}}}
\newcommand{\NormalTok}[1]{\textcolor[rgb]{0.00,0.23,0.31}{#1}}
\newcommand{\OperatorTok}[1]{\textcolor[rgb]{0.37,0.37,0.37}{#1}}
\newcommand{\OtherTok}[1]{\textcolor[rgb]{0.00,0.23,0.31}{#1}}
\newcommand{\PreprocessorTok}[1]{\textcolor[rgb]{0.68,0.00,0.00}{#1}}
\newcommand{\RegionMarkerTok}[1]{\textcolor[rgb]{0.00,0.23,0.31}{#1}}
\newcommand{\SpecialCharTok}[1]{\textcolor[rgb]{0.37,0.37,0.37}{#1}}
\newcommand{\SpecialStringTok}[1]{\textcolor[rgb]{0.13,0.47,0.30}{#1}}
\newcommand{\StringTok}[1]{\textcolor[rgb]{0.13,0.47,0.30}{#1}}
\newcommand{\VariableTok}[1]{\textcolor[rgb]{0.07,0.07,0.07}{#1}}
\newcommand{\VerbatimStringTok}[1]{\textcolor[rgb]{0.13,0.47,0.30}{#1}}
\newcommand{\WarningTok}[1]{\textcolor[rgb]{0.37,0.37,0.37}{\textit{#1}}}

\usepackage{longtable,booktabs,array}
\usepackage{calc} % for calculating minipage widths
% Correct order of tables after \paragraph or \subparagraph
\usepackage{etoolbox}
\makeatletter
\patchcmd\longtable{\par}{\if@noskipsec\mbox{}\fi\par}{}{}
\makeatother
% Allow footnotes in longtable head/foot
\IfFileExists{footnotehyper.sty}{\usepackage{footnotehyper}}{\usepackage{footnote}}
\makesavenoteenv{longtable}
\usepackage{graphicx}
\makeatletter
\newsavebox\pandoc@box
\newcommand*\pandocbounded[1]{% scales image to fit in text height/width
  \sbox\pandoc@box{#1}%
  \Gscale@div\@tempa{\textheight}{\dimexpr\ht\pandoc@box+\dp\pandoc@box\relax}%
  \Gscale@div\@tempb{\linewidth}{\wd\pandoc@box}%
  \ifdim\@tempb\p@<\@tempa\p@\let\@tempa\@tempb\fi% select the smaller of both
  \ifdim\@tempa\p@<\p@\scalebox{\@tempa}{\usebox\pandoc@box}%
  \else\usebox{\pandoc@box}%
  \fi%
}
% Set default figure placement to htbp
\def\fps@figure{htbp}
\makeatother





\setlength{\emergencystretch}{3em} % prevent overfull lines

\providecommand{\tightlist}{%
  \setlength{\itemsep}{0pt}\setlength{\parskip}{0pt}}



 
\usepackage[]{natbib}
\bibliographystyle{plainnat}


% FIX Marginnote duplication
\usepackage{savesym}
\savesymbol{marginfigure}
\savesymbol{marginnote}
\savesymbol{sidenote}

%%%%%%%
%  FIX Makeuppercase error
%  FIX Font clash Math error
%  See https://tex.stackexchange.com/q/202142/157312
% 

\renewcommand{\textls}[2][5]{%
  \begingroup\addfontfeatures{LetterSpace=#1}#2\endgroup
}
\renewcommand{\allcapsspacing}[1]{\textls[15]{#1}}
\renewcommand{\smallcapsspacing}[1]{\textls[10]{#1}}
\renewcommand{\allcaps}[1]{\textls[15]{\MakeTextUppercase{#1}}}
\renewcommand{\smallcaps}[1]{\smallcapsspacing{\scshape\MakeTextLowercase{#1}}}
\renewcommand{\textsc}[1]{\smallcapsspacing{\textsmallcaps{#1}}}

\PassOptionsToPackage{no-math}{fontspec}
% \usepackage[mathlf, minionint,footnotefigures, frenchmath]{MinionPro}
% \setmainfont{$$}
% \setsansfont{TeX Gyre Heros}[Scale=MatchUppercase]

\ExplSyntaxOn
\int_new:N \l_mathcode_minus_int
\int_new:N \l_mathcode_equal_int
\exp_args:Nx \AtBeginDocument {
  \exp_not:n {
    \int_set:Nn \l_mathcode_minus_int { \XeTeXmathcodenum `\- }
    \int_set:Nn \l_mathcode_equal_int { \XeTeXmathcodenum `\= }
  }
  \mathcode \int_eval:n { `\- } = \number \mathcode `\- \scan_stop:
  \mathcode \int_eval:n { `\= } = \number \mathcode `\= \scan_stop:
}
\AtBeginDocument {
  \XeTeXmathcodenum `\- = \l_mathcode_minus_int
  \XeTeXmathcodenum `\= = \l_mathcode_equal_int
}
\ExplSyntaxOff

\usepackage[italic]{mathastext}
% \setromanfont{TeX Gyre Termes}


%%%%%%%


\usepackage{pdfpages}  % for cover page
\graphicspath{{Images/}} % Make Images/ default figure path


\setlength{\parindent}{0pt}%
\setlength{\RaggedRightParindent}{0pt}
\setlength{\JustifyingParindent}{0pt}%
\setlength{\parskip}{\baselineskip}

%%
% Produces a full title page

\renewcommand{\maketitlepage}[0]{%
  \cleardoublepage%
  {%
  \sffamily%
  \begin{fullwidth}%
  \fontsize{12}{14}\selectfont\par\noindent\textcolor{darkgray}{\allcaps{\thanklessauthor}}%
  \vspace{12.5pc}%
  \fontsize{20}{28}\selectfont\par\noindent\textcolor{darkgray}{\allcaps{\thanklesstitle}}%
  \vfill%
  \fontsize{10}{12}\selectfont\par\noindent\allcaps{\thanklesspublisher}%
  \end{fullwidth}%
  }
  \thispagestyle{empty}%
  \clearpage%
}

% DEFINITIONS


% The fancyvrb package lets us customize the formatting of verbatim
% environments.  We use a slightly smaller font.
\usepackage{fancyvrb}
\fvset{fontsize=\normalsize}

%%
% Prints argument within hanging parentheses (i.e., parentheses that take
% up no horizontal space).  Useful in tabular environments.
\newcommand{\hangp}[1]{\makebox[0pt][r]{(}#1\makebox[0pt][l]{)}}

%%
% Prints an asterisk that takes up no horizontal space.
% Useful in tabular environments.
\newcommand{\hangstar}{\makebox[0pt][l]{*}}

%%
% Prints a trailing space in a smart way.
\usepackage{xspace}


% Prints the month name (e.g., January) and the year (e.g., 2008)
\newcommand{\monthyear}{%
  \ifcase\month\or January\or February\or March\or April\or May\or June\or
  July\or August\or September\or October\or November\or
  December\fi\space\number\year
}


% Prints an epigraph and speaker in sans serif, all-caps type.
\newcommand{\epigraph}[2]{%
  \begin{fullwidth}
  \begin{flushright}
  \sffamily\fontsize{8}{10}\selectfont
  \sffamily\footnotesize
  \begin{doublespace}
  \vspace{-8cm}\noindent\allcaps{#1}\\% epigraph
  \noindent\allcaps{#2}\\% author
  \end{doublespace}
  \vspace{5.1cm}
  \end{flushright}
  \end{fullwidth}
  \normalfont
}


\newcommand{\blankpage}{\newpage\hbox{}\thispagestyle{empty}\newpage}


% insert 4cm before quote
\renewenvironment{quote}{
  \list{}{\leftmargin=3.5cm\topsep=0pt}
  \item\relax\small\itshape
}
{\endlist}


%  change chapter formatting
\titlespacing*{\chapter}{0pt}{5cm}{1cm}
% \titlespacing*{\section}{0pt}{.6em}{.3em}
% \titlespacing*{\subsection}{0pt}{.4em}{.2em}

\titlespacing*{\section}{0pt}{0pt}{0pt}
\titlespacing*{\subsection}{0pt}{0pt}{0pt}


%  Change Figure Caption in the Margin size
% \renewenvironment{@tufte@margin@float}[2][-1.2ex]%
%   {\FloatBarrier% process all floats before this point so the figure/table numbers stay in order.
%   \begin{lrbox}{\@tufte@margin@floatbox}%
%   \begin{minipage}{\marginparwidth}%
%     \@tufte@caption@font\footnotesize% <-- Add fontnotesize
%     \def\@captype{#2}%
%     \hbox{}\vspace*{#1}%
%     \@tufte@caption@justification%
%     \@tufte@margin@par%
%     \noindent\normalsize%<-- restored size
%   }
%   {\end{minipage}%
%   \end{lrbox}%
%   \marginpar{\usebox{\@tufte@margin@floatbox}}%
  % }


% \renewcommand\footnotesize{%
%    \@setfontsize\footnotesize\@viiipt{9}%
%    \abovedisplayskip 5\p@ \@plus2\p@ \@minus4\p@
%    \abovedisplayshortskip \z@ \@plus\p@
%    \belowdisplayshortskip 2.8\p@ \@plus\p@ \@minus2\p@
%    \def\@listi{\leftmargin\leftmargini
%                \topsep 2.5\p@ \@plus\p@ \@minus\p@
%                \parsep 2\p@ \@plus\p@ \@minus\p@
%                \itemsep \parsep}%
%    \belowdisplayskip \abovedisplayskip
% }

% % Define Tuftian float styles (with the caption in the margin)
% \newcommand{\floatc@tufteplain}[2]{%
% \begin{lrbox}{\@tufte@caption@box}%
%   \begin{minipage}[\floatalignment]{\marginparwidth}\hbox{}%
%     \footnotesize\@tufte@caption@font{\@fs@cfont #1:} #2\par\normalsize%
%   \end{minipage}%
% \end{lrbox}%
% \smash{\hspace{\@tufte@caption@fill}\usebox{\@tufte@caption@box}}%
% }
\makeatletter
\@ifpackageloaded{bookmark}{}{\usepackage{bookmark}}
\makeatother
\makeatletter
\@ifpackageloaded{caption}{}{\usepackage{caption}}
\AtBeginDocument{%
\ifdefined\contentsname
  \renewcommand*\contentsname{Table of contents}
\else
  \newcommand\contentsname{Table of contents}
\fi
\ifdefined\listfigurename
  \renewcommand*\listfigurename{List of Figures}
\else
  \newcommand\listfigurename{List of Figures}
\fi
\ifdefined\listtablename
  \renewcommand*\listtablename{List of Tables}
\else
  \newcommand\listtablename{List of Tables}
\fi
\ifdefined\figurename
  \renewcommand*\figurename{Figure}
\else
  \newcommand\figurename{Figure}
\fi
\ifdefined\tablename
  \renewcommand*\tablename{Table}
\else
  \newcommand\tablename{Table}
\fi
}
\@ifpackageloaded{float}{}{\usepackage{float}}
\floatstyle{ruled}
\@ifundefined{c@chapter}{\newfloat{codelisting}{h}{lop}}{\newfloat{codelisting}{h}{lop}[chapter]}
\floatname{codelisting}{Listing}
\newcommand*\listoflistings{\listof{codelisting}{List of Listings}}
\makeatother
\makeatletter
\makeatother
\makeatletter
\@ifpackageloaded{caption}{}{\usepackage{caption}}
\@ifpackageloaded{subcaption}{}{\usepackage{subcaption}}
\makeatother
\makeatletter
\@ifpackageloaded{sidenotes}{}{\usepackage{sidenotes}}
\@ifpackageloaded{marginnote}{}{\usepackage{marginnote}}
\makeatother
\makeatletter
\@ifpackageloaded{bibentry}{}{\usepackage{bibentry}}
\@ifpackageloaded{marginfix}{}{\usepackage{marginfix}}
\makeatother
\makeatletter
\nobibliography*
\makeatother
\usepackage{bookmark}
\IfFileExists{xurl.sty}{\usepackage{xurl}}{} % add URL line breaks if available
\urlstyle{same}
\hypersetup{
  pdftitle={Tufte-Quarto Project Type Documentation},
  pdfauthor={Fred Guth},
  colorlinks=true,
  linkcolor={blue},
  filecolor={Maroon},
  citecolor={Blue},
  urlcolor={Blue},
  pdfcreator={LaTeX via pandoc}}



%  TITLE PAGE

\title[Tufte-Quarto]{Tufte-Quarto Project Type\\
Documentation} 
\author{Fred Guth} 
\publisher{The Publisher}





\begin{document}
\IfFileExists{Images/bookcover.pdf}{\includepdf{Images/bookcover.pdf}\clearpage}{}

\frontmatter
\maketitle
\pagenumbering{gobble} 
\IfFileExists{01-Front/copyright.tex}{\input{01-Front/copyright.tex}\clearpage}{}
\IfFileExists{01-Front/dedication.tex}{\input{01-Front/dedication.tex}\clearpage}{}
\IfFileExists{01-Front/epigraph.tex}{\input{01-Front/epigraph.tex}\clearpage}{}
\pagenumbering{arabic}
\renewcommand*\contentsname{Contents}
{
\hypersetup{linkcolor=}
\setcounter{tocdepth}{1}
\tableofcontents
}

\mainmatter
\bookmarksetup{startatroot}

\chapter{Preface}\label{preface}

\begin{figure}[H]

{\centering \pandocbounded{\includegraphics[keepaspectratio]{Images/et_midjourney_transparent.png}}

}

\caption{A portrait Edward R. Tufte, godfather of charts. Art by Fred
Guth and MidJourney.}

\end{figure}%

The Tufte-Quarto project is an homage to \hyperref[sec-tufte]{Edward
Tufte}. It simplifies the production of books using a layout that
resembles \href{https://www.edwardtufte.com/tufte/books_vdqi}{his
books}\citep{Tufte:vdqi, Tufte:en, Tufte:ve}\marginpar{\begin{footnotesize}\bibentry{Tufte:vdqi}\vspace{2mm}\par\end{footnotesize}}\marginpar{\begin{footnotesize}\bibentry{Tufte:en}\vspace{2mm}\par\end{footnotesize}}\marginpar{\begin{footnotesize}\bibentry{Tufte:ve}\vspace{2mm}\par\end{footnotesize}}
with a companion website. The goal is for the book and website share the
same aesthetics.

\section{Showcase}\label{showcase}

\subsection{This documentation}\label{this-documentation}

This documention itself is using Tufte-Quarto.

\subsection{Dissertation}\label{dissertation}

As an example, \href{https://fredguth.github.io/IBToDL}{my dissertation}
is being refactored with Tufte-Quarto, it gives an idea of a complex
document built with Tufte-Quarto.

\section{Usage}\label{usage}

To use this project type just:

\texttt{quarto\ use\ template\ fredguth/tufte-quarto}

\phantomsection\label{sec-tufte}
\section{About Edward R. Tufte}\label{about-edward-r.-tufte}

Professor Emeritus of Political Science, Statistics and Computer Sciente
at Yale University, Edward Tufte is an expert in the presentation of
informational. Also known as \emph{the godfather of charts}. Check
\href{https://www.edwardtufte.com/tufte/}{his website}.

\bookmarksetup{startatroot}

\chapter{Customizing}\label{customizing}

You can customize your book by changing \texttt{\_quarto.yml} and the
files it referendes. This project organize files in 4 basic folders:

\begin{itemize}
\tightlist
\item
  01-Front: for frontmatter content;
\item
  02-Main: for mainmatter content;
\item
  03-Back: for backmatter content;
\end{itemize}

\section{Book Structure}\label{book-structure}

\begin{itemize}
\tightlist
\item
  Cover
\item
  Front Matter

  \begin{itemize}
  \tightlist
  \item
    Title Page
  \item
    Copyright
  \item
    Dedication
  \item
    Epigraph/dictum
  \item
    Table of Contents
  \end{itemize}
\item
  Main Matter

  \begin{itemize}
  \tightlist
  \item
    chapters
  \end{itemize}
\item
  Back Matter

  \begin{itemize}
  \tightlist
  \item
    references
  \item
    appendices
  \item
    acknowledgements
  \item
    colophon
  \end{itemize}
\end{itemize}

\section{File Structure}\label{file-structure}

\begin{longtable}[]{@{}
  >{\raggedright\arraybackslash}p{(\linewidth - 4\tabcolsep) * \real{0.1930}}
  >{\raggedright\arraybackslash}p{(\linewidth - 4\tabcolsep) * \real{0.4561}}
  >{\raggedright\arraybackslash}p{(\linewidth - 4\tabcolsep) * \real{0.3509}}@{}}
\toprule\noalign{}
\begin{minipage}[b]{\linewidth}\raggedright
Book Part
\end{minipage} & \begin{minipage}[b]{\linewidth}\raggedright
File
\end{minipage} & \begin{minipage}[b]{\linewidth}\raggedright
How to customize
\end{minipage} \\
\midrule\noalign{}
\endhead
\bottomrule\noalign{}
\endlastfoot
Cover & images/bookcover.pdf & Replace the file. If no file, no
cover. \\
Titlepage & \_quarto.yml & Built from metadata. For further
customization use
\href{https://quarto.org/docs/journals/templates.html}{partials}. \\
Copyright & 01-Front/dedication.tex & Replace or exclude/rename file. \\
Dedication & 01-Front/dedication.tex & Replace or exclude/rename
file. \\
Epigraph & 01-Front/epigraph.tex & Replace or exclude/rename file. \\
ToC & \_quarto.yml & Built from metadata. For further customization use
\href{https://quarto.org/docs/journals/templates.html}{partials}. \\
Chapters & \_quarto.yml & You can reference
\href{https://quarto.org/docs/books/book-structure.html}{chapters, parts
and apendices}. \\
Acknolegements & 03-Back/ack.qmd & Replace or exclude/rename file. \\
Colophon & 03-Back/colophon.qmd & Replace or exclude/rename file. \\
\end{longtable}

\subsection{Advanced Cutomization}\label{advanced-cutomization}

In \texttt{\_extensions/tufte/\_extension.yml} you can check the default
settings of the Tufte-Quarto Book type. All these settings can be
overwritten in \_quarto.yml.

The project uses
\href{https://tufte-latex.github.io/tufte-latex/}{tufte-book} Latex
class.

\section{Customizing the Website}\label{customizing-the-website}

Besides the settings from
\texttt{\_extensions/tufte/\_extension.yml\ \textgreater{}\ format\ \textgreater{}\ html}
that can be overwritten in
\texttt{\_quarto.yml\ \textgreater{}\ format\ \textgreater{}\ html}, you
can further customize the website by creating a \texttt{style.css} file
and referencing it in \texttt{\_quarto.yml}. Check
\texttt{\_extensions/tufte/style.css} as an example, but avoid changing
directly there as Tufte-Quarto updates may break your changes.

\bookmarksetup{startatroot}

\chapter{Known-issues}\label{known-issues}

\begin{enumerate}
\def\labelenumi{\arabic{enumi}.}
\item
  Tufte-book can't handle label inside caption.

  Tufte-book.cls breaks when processing the line bellow:

\begin{Shaded}
\begin{Highlighting}[]
\NormalTok{![A way of flying](/Images/goya.jpg)\{.column{-}body \#fig{-}goya\}}
\end{Highlighting}
\end{Shaded}

  which becomes:

\begin{Shaded}
\begin{Highlighting}[]
\KeywordTok{\textbackslash{}begin}\NormalTok{\{}\ExtensionTok{figure}\NormalTok{\}}

\NormalTok{\{}\FunctionTok{\textbackslash{}centering} \BuiltInTok{\textbackslash{}includegraphics}\NormalTok{\{}\ExtensionTok{Images/goya.jpg}\NormalTok{\}}

\NormalTok{\}}

\FunctionTok{\textbackslash{}caption}\NormalTok{\{}\KeywordTok{\textbackslash{}label}\NormalTok{\{}\ExtensionTok{fig{-}goya}\NormalTok{\}A way of flying\}}

\KeywordTok{\textbackslash{}end}\NormalTok{\{}\ExtensionTok{figure}\NormalTok{\}}
\end{Highlighting}
\end{Shaded}
\item
  Can't render svg image in pdf and can't render pdf image in html.

  \begin{itemize}
  \tightlist
  \item
    current solution is quite ugly:
  \end{itemize}

\begin{Shaded}
\begin{Highlighting}[]
\NormalTok{::: \{.content{-}hidden unless{-}format="pdf" \}}

\NormalTok{![IBT "genealogy" tree.](/Images/dissertation{-}map.pdf)\{.column{-}margin width=90\%\}}

\NormalTok{:::}

\NormalTok{::: \{.content{-}hidden unless{-}format="html"\}}

\NormalTok{![IBT "genealogy" tree.](/Images/dissertation{-}map.svg)\{.column{-}margin width=90\%\}}

\NormalTok{:::}
\end{Highlighting}
\end{Shaded}

  For avoiding duplicating code, files are .svg in html by defaul and
  .pdf in pdf. So you can only:

\begin{Shaded}
\begin{Highlighting}[]

\NormalTok{![IBT "genealogy" tree.](/Images/dissertation{-}map)\{.column{-}margin width=90\%\}}
\end{Highlighting}
\end{Shaded}
\item
  Tufte-class works only until subsection -\textgreater{}
  \texttt{\#\#\#\ subsection};
  \texttt{\#\#\#\#\ subsubsection}-\textgreater{} returns error
\item
  sidecite is duplicating citations in the same margin. Solved this same
  problem in my dissertation and in the kaobook class. Only problem is
  that Tufte-book class is a little too cryptic for me.
\end{enumerate}

\part{About}

\chapter*{References}\label{references}

\markboth{References}{References}

\renewcommand{\bibsection}{}
\bibliography{references.bib}

\chapter*{Acknowledgements}\label{acknowledgements}

\markboth{Acknowledgements}{Acknowledgements}

Many thanks to the Quarto team for creating this wonderful tool. I am
also grateful for the support I got at
\href{https://github.com/quarto-dev/quarto-cli/discussions?discussions_q=author\%3Afredguth}{Quarto's
Discussion Board}, specialy:

\begin{itemize}
\tightlist
\item
  Mickaël Canouil (\texttt{@mcanouil});
\item
  Charles Teague (\texttt{@dragonstyle});
\item
  Raniere Silva (\texttt{@rgaiacs}); and
\item
  Christophe Dervieux (\texttt{@cderv})
\end{itemize}

\chapter*{Colophon}\label{colophon}

\markboth{Colophon}{Colophon}

Composed in ET Book, a free and open-sopurce typeface designed by Dmitry
Krasny, Bonnie Scranton, and Edward Tufte; and \TeX  Gyre Heros, a
fontbased on the URW Nimbus Sans L kindly released by URW++ Design and
Development Inc.~under GFL. Based on
\href{https://tufte-latex.github.io/tufte-latex/}{Tufte-Latex} and
\href{https://quarto.org}{Quarto}.

{\marginnote{\begin{footnotesize}\pandocbounded{\includegraphics[keepaspectratio]{Images/gutemberg_press.pdf}}\end{footnotesize}}}


\backmatter





\end{document}
